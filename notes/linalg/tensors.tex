\documentclass[a4paper,14pt]{article}
\title{Tensor Products of Vector Spaces}
\usepackage{amsmath, amssymb, amsthm}
\usepackage{tikz-cd}
\newtheorem*{prop}{Proposition}
\newtheorem*{corollary}{Corollary}
\newtheorem*{remark}{Remark}
\newtheorem*{defn}{Definition}
\newtheorem*{thm}{Theorem}
\begin{document}
\maketitle
\section{Hom spaces, dual spaces}
If $V$ and $W$ are two vector spaces over the same field $\mathbb{F}$, then the set of all linear maps $T: V \to W$ is denoted by $Hom(V, W)$. Under the usual definition of function addition and scalar multiplication, $Hom(V, W)$ is a vector space.

Since any field $\mathbb{F}$ can be considered as a vector space over itself, we can in particular form the space $Hom(V, \mathbb{F})$. This space is called the \textbf{dual space} of $V$, and will be denoted $V^{\ast}$.

\section{Quotient spaces}
If $V$ is a vector space over $\mathbb{F}$ and $W$ is a subspace of $V$, a \textbf{coset} of $W$ is a set $v + W := \{v + w : w \in W\}$. 

\begin{prop}
The collection of cosets is a partition of $V$.
\begin{proof}
If $z$ is in both $u + W$ and $v + W$, then $u + w_1 = z = v + w_2$, so $v = u + (w_1 - w_2) \in u + W$. Similarly, $u \in v + W$. If $y \in u + W$, then $y - u \in W$, so $y \in v + W$. Similarly every $z \in v + W$ is in $u + W$, so $u + W = v + W$.
\end{proof}
\end{prop}

\begin{prop}
For any vector space $V$ and subspace $W$, you can define addition between cosets and scalar multiplication on cosets such that collection of cosets becomes a vector space over $\mathbb{F}$.
\end{prop}

\begin{remark}
The vector space of cosets of $W$ in $V$ is called the \textbf{quotient space}, and is denoted $V/W$.
\end{remark}

\begin{prop}
The map $\pi: V \to V/W$ defined by $v \mapsto v + W$ is a linear map.
\end{prop}

\begin{remark}
The function $\pi$ is called the \textbf{canonical map}.
\end{remark}

\begin{prop}
The kernel of any linear map $f: V \to W$ is a subspace of $V$, so $V / ker f$ is a quotient space.
\end{prop}

\begin{thm}
If $f: V \to W$ is linear and $U$ is a subspace of $V$ such that $U \subseteq ker f$, then there is a unique linear map $\phi: V / U \to W$ such that $\pi ; \phi = f$, where $\pi$ is the canonical map.

\begin{proof}
    Define $\phi$ by $v + U \mapsto vf$. Routine verification shows that it's well-defined, linear and satisfies $\pi ; \phi = f$. $\phi$ is unique because any other candidate $\psi$ must have $(v + U)\phi = vf = (v \pi) \psi = (v + U) \psi$, meaning $\phi = \psi$.
\end{proof}
\end{thm}

\section{Vector space of scalar-valued functions}
\begin{prop}
If $\mathbb{F}$ is any field and $X$ is any set, you can form a vector space by considering the collection of all functions $X \to \mathbb{F}$. 

\begin{proof}
If $f, g: X \to \mathbb{F}$, we define addition by $x(f + g) := xf + xg$ and scalar multiplication by $x(\alpha f) = \alpha (x f)$. That this is a vector space is a routine verification.
\end{proof}
\end{prop}

\begin{remark}
This vector space is denoted $\mathbb{F}^X$.

You are probably familiar with the vector space $\mathbb{F}^n$. If you think about it, $\mathbb{F}^n$ is actually a special case of a space $\mathbb{F}^X$, where $X = \{1, \ldots, n\}$. The general idea, then, is that a function $X \to \mathbb{F}$ is a bag of scalars labeled by elements of $X$ There is exactly one scalar for each label. We can add two bags $B_1$ and $B_2$ of labeled scalars: the result is another bag of labeled scalars where for each label $x \in X$, the associated scalar is the sum of the scalars labeled by $x$ in $B_1$ and $B_2$. Scalar multiplication of a bag $B$ by $\alpha$ is just the bag where each scalar in $B$ is multiplied by $\alpha$.
\end{remark}

\begin{defn}
A function $f: X \to \mathbb{F}$ is said to have \textbf{finite support} when the set $\{x \in X : xf \neq 0\}$ is finite.
\end{defn}

\begin{prop}
The subset of $\mathbb{F}^X$ of functions with finite support is a subspace of $\mathbb{F}^X$.
\end{prop}

\section{Free vector spaces}
\begin{defn}
For any set $X$, a \textbf{free vector space} on $X$ with respect to some field $\mathbb{F}$, is any pair $(V, f)$ where $V$ is a vector space on $\mathbb{F}$ and $f: X \to V$ is a function such that for any vector space $W$ over $\mathbb{F}$ and any function $g: X \to W$, there is a unique linear map $T: V \to W$ such that $f;T = g$.

In other words, for every $W$ and every function $g: X \to W$ there is a unique linear $T: V \to W$ such that this diagram commutes:

\begin{center}
\begin{tikzcd}[row sep=large]
    X \arrow[rd, "g"] \arrow[r, "f"] 
    & V \arrow[d, dashed, "\exists! \> T"] \\
                                   & W
\end{tikzcd}
\end{center}

This property is called the "universal property" for free vector spaces.
\end{defn}

\begin{prop}
    If a free vector space of a set $X$ over a field $\mathbb{F}$ exists, it is ``unique up to unique isomorphism'': if $(V, f: X \to V)$ and $(W, g: X \to W)$ are free vector spaces for $X$ over $\mathbb{F}$, there is a unique linear isomorphism $T: V \to W$ such that $f;T = g$.

\begin{proof}
    Since $(V, f)$ is a free vector space, by the universal property there is a linear map $r: V \to W$ such that $f;r = g$. Similarly there is a linear map $s: W \to V$ such that $g;s = f$. This implies $g;s;r = g$ and $f;r;s = f$.

\begin{center}
\begin{tikzcd}[row sep=large]
    & X \arrow[ld, "f"swap] \arrow[rd, "g"] \\
    V \arrow[rr, yshift=0.7ex, "r"] & & W \arrow[ll, yshift=-0.7ex, "s"]
\end{tikzcd}
\end{center}

 But we can apply the universal properties to $f$ and $g$ themselves: there is a unique linear map $p: V \to V$ such that $f = f;p$, and similarly a unique linear map $q: W \to W$ such that $g = g;q$. Since the identity maps $id_V$ and $id_W$ on $V$ and $W$ work for $p$ and $q$, respectively, and since $s;r$ and $r;s$ also fit the criteria, we must have $r;s = id_V$ and $s;r = id_W$. So $r$ and $s$ are inverses of each other, hence bijective, hence linear isomorphisms.

To prove that $r$ is the unique linear isomorphism $L: V \to W$ such that $f;L = g$, note that $f = g;s$, so $g;s;L = g$. $s;L$, being the composition of linear maps, is linear, so we must have $s;L = id_W$. So $L$ is a post-inverse for $s$, implying $L = r$.
\end{proof}
\end{prop}

\begin{prop}
There is a free vector space for every set $X$ and every field $\mathbb{F}$.
\begin{proof}
    The pair $(V, f)$ works, where $V$ is the subspace of $\mathbb{F}^X$ of functions of finite support and $f: X \to V$ is defined by $xf$ being the function that maps $x$ to $1 \in \mathbb{F}$ and $y \neq x$ to $0 \in \mathbb{F}$.
\end{proof}
\end{prop}

\begin{prop}
If $(V, f)$ is a free vector space for $X$, then $f$ is injective.
\begin{proof}
If $xf = yf$ for some $x \neq y \in X$, then any function $g: X \to W$ which has $xg \neq yg$ will not have a factorization through $f$, contradicting the universal property.
\end{proof}
\end{prop}

\section{Multilinear maps}
If $V_1, \ldots, V_n$ are some vector spaces over a field $\mathbb{F}$, then for any integer $i$, $1 \leq i \leq n$, and for any $v = (v_1, \ldots, v_{i-1}, v_{i+1}, \ldots, v_n) \in \prod_{j=1, j \neq i}^n V_j$, we can define the function $\lambda_i[v]: V_i \to \prod_1^n V_j$ by $z \lambda_i[v] := (v_1, \ldots, v_{i-1}, z, v_{i+1}, \ldots, v_n)$. It is routine to prove that each such function is a linear map $V_i \to \prod_1^n V_j$.

For any vector space $W$ over $\mathbb{F}$, a function $\phi: \prod_1^n V_j$ is said to be \textbf{multilinear} when for every $i$ and every $v \in \prod_{j=1, j \neq i}^n V_j$, the composite function $\lambda_i[v];\phi$ is a linear map $V_i \to W$. When $W = \mathbb{F}$ (that is, the vector space over $\mathbb{F}$, then $\phi$ is said to be a \textbf{multilinear form}.

\begin{prop}
Denote the collection of all multilinear maps $\prod_1^n V_i \to W$ by $ML(V_1, \ldots, V_n; W)$. By the usual definition of function addition and function scalar multiplication, this is a vector space over $\mathbb{F}$.
\end{prop}

\section{Tensor products}
A tensor product of vector spaces $(V_1, \ldots, V_n)$ is a vector space $Z$ and a multilinear map $\otimes$ such that $\otimes$ is, in some sense, the most general multilinear map on $\prod_1^n V_i$. We give a universal property that defines the tensor product (similar to what was done for free vector spaces), prove some general properties about it, and then explicitly construct one of the tensor products.

\begin{defn}
    For any vector spaces $V_1, \ldots, V_n$ over a common field $\mathbb{F}$, a \textbf{tensor product} of $(V_1, \ldots, V_n)$ is a pair $(Z, \otimes)$ where $Z$ is a vector space and $\otimes$ is a multilinear map $\prod_1^n V_i \to Z$ such that for every vector space $W$ over $\mathbb{F}$ and every multilinear map $f: \prod_1^n V_i \to W$, there is a unique linear map $T: Z \to W$ such that $f = \otimes;T$.

In other words, for every vector space $W$ and every multilinear map $f: \prod_1^n V_i \to W$ there is a unique linear $T: Z \to W$ such that this diagram commutes:

\begin{center}
\begin{tikzcd}[row sep=large]
    \prod_1^n V_i \arrow[rd, "f"] \arrow[r, "\otimes"]
    & Z \arrow[d, dashed, "\exists! \> T"] \\
    & W
\end{tikzcd}
\end{center}
\end{defn}

\begin{prop}
If $(Z, \otimes)$ is a tensor product of $(V_1, \ldots, V_n)$, then it is unique up to unique isomorphism.
\begin{proof}
This has the same form as the "unique up to unique isomorphism" proof for free vector spaces, so it is omitted here.
\end{proof}
\end{prop}

\begin{prop}
If $(Z, \otimes)$ is a tensor product of $(V_1, \ldots, V_n)$ and $W$ is any vector space over the same field as the $V_i$'s, then $Hom(Z, W)$ is isomorphic to $ML(V_1, \ldots, V_n; W)$.
\begin{proof}
    The universal property induces a map $\phi: ML(V_1, \ldots, V_n; W) \to Z$. If $T: Z \to W$ is any linear map, then $\otimes;T$ is multilinear since $\otimes$ is, so $(\otimes;T) \phi = T$, hence $\phi$ is surjective. It must also be injective: if $T = f \phi = g \phi$, then $f = \otimes;T = g$.
\end{proof}
\end{prop}
\begin{corollary}
For any tensor product $(Z, \otimes)$ of $(V_1, \ldots, V_n)$, the space of all multilinear forms on $\prod_1^n V_i$ is isomorphic to $Z^{\ast}$.
\end{corollary}

\begin{prop}
If $(Z, \otimes)$ is a tensor product of $(V_1, \ldots, V_n)$, then $img \otimes$ generates $Z$.
\begin{proof}
Let $Y$ be the subspace of $Z$ generated by $img \otimes$. Consider $\otimes': \prod_1^n V_i \to Y$, defined to be $\otimes$ but with its codomain restricted to $Y$. $\otimes'$ is clearly multilinear since $\otimes$ is, so we have some $T: Z \to Y$, linear, such that $\otimes; T = \otimes'$. But also, if we let $j: Y \to Z$ be the inclusion of $Y$ into $Z$, it is linear and $\otimes'; j = \otimes$. 

\begin{center}
\begin{tikzcd}
    \prod_1^n V_i \arrow[rd, "\otimes'"] \arrow[r, "\otimes"] \arrow[rdd, "\otimes"]
    & Z \arrow[d, "T"] \arrow[dd, bend left=45, "id_Z"] \\
    & Y \arrow[d, hook, "j"] \\
    & Z
\end{tikzcd}
\end{center}

By combining these two, we get: $\otimes;T;j = \otimes$. But $T;j$ is linear, and by the universal property $id_Z$ is the unique linear map such that $\otimes; id_Z = \otimes$. So $T;j = id_Z$, implying $j$ is surjective. So $Y = Z$.

\end{proof}
\end{prop}

\begin{prop}
For any vector spaces $(V_1, \ldots, V_n)$ over a common field $\mathbb{F}$, there is a tensor product. The vector space is denoted $\bigotimes_1^n V_i$ (or $V_1 \otimes V_2$ in the case of $n = 2$) and the multilinear map is denoted $\otimes: \prod_1^n V_i \to \bigotimes_1^n V_i$, where usually infix notation is used: for all $(v_1, \ldots, v_n) \in \prod_1^n V_i$, $(v_1, \ldots, v_n) \mapsto v_1 \otimes \cdots \otimes v_n$.
\begin{proof}
    Let $(F(\prod_1^n V_i), \iota)$ be any free vector space on $\prod_1^n V_i$. If $W$ is any vector space and $\phi: \prod_1^n V_i \to W$ is any multilinear map, by the universal property of the free vector space we have a unique linear map $T: F(\prod_1^n V_i) \to W$ such that $\iota;T = \phi$.

\begin{center}
\begin{tikzcd}
    \prod_1^n V_i \arrow[d, "\phi"] \arrow[r, "\iota"]
    & F(\prod_1^n V_i) \arrow[dl, "T"] \\
    W
\end{tikzcd}
\end{center}

For all $(v_1, \ldots, v_n) \in \prod_1^n V_i$, let us denote $(v_1, \ldots, v_n) \iota$ by $[v_1, \ldots, v_n]$. $\iota$ is not multilinear since the zero of $F(\prod_1^n V_i)$ is not in the image of $\iota$, but we can make it multilinear by composing it with another map.

Consider the subspace $Z$ generated by all vectors of the form:

$$[v_1, \dots, v_{i-1}, \alpha v, v_{i+1}, \dots, v_n] - [v_1, \dots, v_{i-1}, v, v_{i+1}, \dots, v_n]$$

and

$$[v_1, \dots, v_{i-1}, v + w, v_{i+1}, \dots, v_n] - [v_1, \dots, v_{i-1}, v, v_{i+1}, \dots, v_n] - [v_1, \dots, v_{i-1}, w, v_{i+1}, \dots, v_n]$$

for some $i$, $1 \leq i \leq n$ and some $\alpha \in \mathbb{F}$, $v, w \in V_i$. If we let $\pi: F(\prod_1^n V_i) \to F(\prod_1^n V_i) / Z$ be the canonical map from $F(\prod_1^n V_i)$ to the quotient space induced by $Z$, the claim is that $(F(\prod_1^n V_i), \iota; \pi)$ is a tensor product. If we denote $(v_1, \dots, v_n) \iota \pi$ by $\langle v_1, \dots, v_n \rangle$, then, for example, $\langle v + w, v_2, \dots, v_n \rangle = \langle v, v_2, \dots, v_n \rangle + \langle w, v_2, \dots, v_n \rangle$ since the difference of the two sides is an element of $Z$. So $\iota ; \pi$ is multilinear.

Now, $Z$ is a subset of $ker T$ because $\phi$ is multilinear. Also, by the property of the quotient space, there is a unique linear $f: F(\prod_1^n V_i) / Z \to W$ such that $\pi; f = T$.

\begin{center}
\begin{tikzcd}
    \prod_1^n V_i \arrow[d, "\phi"] \arrow[r, "\iota"]
    & F(\prod_1^n V_i) \arrow[dl, "T"] \arrow[d, "\pi"] \\
    W & F(\prod_1^n V_i) / Z \arrow[l, "f"]
\end{tikzcd}
\end{center}

So $(F(\prod_1^n V_i)/Z, \iota;\pi)$ is a tensor product for $(V_1, \ldots, V_n)$.
\end{proof}
\end{prop}

\begin{prop}
For any vector spaces $X, Y, Z$ over a common field, we have $(X \otimes Y) \otimes Z$ is isomorphic to $X \otimes Y \otimes Z$ is isomorphic to $X \otimes (Y \otimes Z)$.
\begin{proof}
    We will construct the isomorphism between $(X \otimes Y) \otimes Z$ and $X \otimes Y \otimes Z$ and wave our hands about the other isomorphism. First, for any $z \in Z$, we can construct a map $\phi_z: X \times Y \to X \otimes Y \otimes Z$ by $(x, y) \phi_z := x \otimes y \otimes z$. This function is multilinear, so by the universal property of $X \otimes Y$, there's a unique linear map $T_z: X \otimes Y \to X \otimes Y \otimes Z$ such that $(x \otimes y ) T_z = x \otimes y \otimes z$ for all $(x, y) \in X \times Y$.

\begin{center}
\begin{tikzcd}[row sep=large]
    X \times Y \arrow[rd, "\phi_z"] \arrow[r, "\otimes"] 
    & X \otimes Y \arrow[d, "T_z"] \\
    & X \otimes Y \otimes Z
\end{tikzcd}
\end{center}

Using $T_z$, we can define a function $\phi: (X \otimes Y) \times Z \to X \otimes Y \otimes Z$ by $(a, z) \phi := a T_z$. This is linear in the first parameter because each $T_z$ is linear. Also, 

\begin{align*}
(x \otimes y) T_{z_1 + z_2} & = x \otimes y \otimes (z_1 + z_2) \\
                            & = x \otimes y \otimes z_1 + x \otimes y \otimes z_2 \\
                            & = (x \otimes y) T_{z_1} + (x \otimes y) T_{z_2}
\end{align*}
    
for all $x \in X$, $y \in Y$. Since elements $x \otimes y$ generate $X \otimes Y$, we have $T_{z_1 + z_2} = T_{z_1} + T_{z_2}$. For a similar reason, $T_{\alpha z} = \alpha T_z$ for all $\alpha \in \mathbb{F}$. So $\phi$ is bilinear, which gives, by the universal property, a linear map $T: (X \otimes Y) \otimes Z \to X \otimes Y \otimes Z$.

\begin{center}
\begin{tikzcd}[row sep=large]
    (X \otimes Y) \times Z \arrow[rd, "\phi"] \arrow[r, "\chi"] 
    & (X \otimes Y) \otimes Z \arrow[d, "T"] \\
                                   & X \otimes Y \otimes Z
\end{tikzcd}
\end{center}

Now, if we define a map $\psi: X \times Y \times Z \to (X \otimes Y) \otimes Z$ by $(x, y, z) \psi := (x \otimes y) \otimes z$, this is a multilinear map, so there's an induced linear map $S: X \otimes Y \otimes Z \to (X \otimes Y) \otimes Z$. By the universal property, this map $S$ has that $(x \otimes y \otimes z) ST = x \otimes y \otimes z$ for all $(x, y, z) \in X \times Y \times Z$. So $ST$ is a linear map $X \otimes Y \otimes Z \to X \otimes Y \otimes Z$ such that, when it is restricted to the generating set of $\{x \otimes y \otimes z : (x, y, z) \in X \times Y \times Z\}$, becomes the identity map. So in fact $ST$ is the identity map.

Similarly, it can be proved that the elements of the form $(x \otimes y) \otimes z$ are a generating set for $(X \otimes Y) \otimes Z$ (elements $a \otimes z$ for $a \in X \otimes Y$ definitely are, but $x \otimes y$ generate $X \otimes Y$ and $\otimes$ is bilinear), so for a similar reason $TS$ is an identity map as well. So $S$ and $T$ are linear and inverses of one another, meaning that $X \otimes Y \otimes Z$ and $(X \otimes Y) \otimes Z$ are isomorphic.
\end{proof}
\end{prop}

\begin{prop}
For any spaces $V$ and $W$ over a field $\mathbb{F}$, $V \otimes W$ is isomorphic to $W \otimes V$.
\begin{proof}
    The multilinear maps $\phi: V \times W \to W \otimes V$ and $\psi: W \times V \to V \otimes W$ defined by $(v, w) \phi := w \otimes v$ and $(w, v) \psi := v \otimes w$ induce, respectively, linear maps $S: V \otimes W \to W \otimes V$ and $T: W \otimes V \to V \otimes W$. We have that $(v \otimes w) ST = v \otimes w$ and $(w \otimes v) TS = w \otimes v$ for all $v \in V$, $w \in W$. Since these generate $V \otimes W$ and $W \otimes V$, $S$ and $T$ are isomorphisms.
\end{proof}
\end{prop}

\section{Finite dimensional facts}

\begin{prop}
    If $V_1, \ldots, V_n$ are all finite dimensional vector spaces with $V_i$ having dimension $d_i$, and, for all $i$, $\beta_i = \{v_{i,1}, \dots, v_{i,d_i}\}$ is a basis for $V_i$, then if $f: \prod_1^n \beta_i \to \mathbb{F}$ is any function, $f$ can be extended to exactly one multilinear form $g: \prod_1^n V_i \to \mathbb{F}$.

Furthermore, $dim ML(V_1, \ldots, V_n; \mathbb{F}) = \prod_1^n dim V_i$.

\begin{proof}
    If $\phi: \prod_1^n V_i \to \mathbb{F}$ is any multilinear form, then for all $z \in \prod_1^n V_i$, $z = (z_1, \dots, z_n) = (\sum_j \alpha_{1,j} v_{1,j}, \dots, \sum_j \alpha_{n,j} v_{n,j})$, so
    
$$z \phi = \sum_{j \in \prod_1^n [d_i]} \alpha_{1,j_1} \cdots \alpha_{n,j_n} (v_{1,j_1}, \ldots, v_{n,j_n}) \phi$$

where $[n] := \{1, \ldots, n\}$ for any $n$. This shows that any multilinear map is completely determined by how the "basis tuples" $(v_{1,j_1}, \ldots, v_{n,j_n})$ get mapped, which proves the first statement.

For any $I \in \prod_1^n [d_i]$ we define $f_I: \prod_1^n V_i \to \mathbb{F}$ to be the multilinear form that maps the basis tuple $(v_{I_1}, \ldots, v_{I_n})$ to $1$ and all other basis tuples to $0$. Then the collection of these functions is a basis for $ML(V_1, \ldots, V_n; \mathbb{F})$.
\end{proof}
\end{prop}

\begin{prop}
If $V_1, \ldots, V_n$ are all finite dimensional vector spaces, then $dim(\otimes_1^n V_i) = \prod_1^n dim V_i$.
\begin{proof}
The previous proposition establishes that $dim ML(V_1, \ldots, V_n; \mathbb{F}) = \prod_1^n dim V_i$. But we know that $ML(V_1, \ldots, V_n; \mathbb{F})$ and $(\otimes_1^n V_i)^{\ast} = Hom(\otimes_1^n V_i, \mathbb{F}$ are isomorphic. If we could establish that $\otimes_1^n V_i$ is finite-dimensional, then the result would be proved since every finite dimensional vector space is isomorphic with its dual.

But we know the set of elements $v_1 \otimes \dots \otimes v_n$ generates $\otimes_1^n V_i$, and we know that $v_1 \otimes \dots \otimes v_n$ is a linear combination of "basis tensors" $e_1 \otimes \dots \otimes e_n$, where each $e_i$ is an element of a basis $\beta_i$ for $V_i$. So the collection of "basis tensors" generates $\otimes_1^n V_i$, hence it is finitely-generated, so finite dimensional.
\end{proof}
\end{prop}


\begin{prop}
If $V$ and $W$ are finite dimensional, then $Hom(V, W) \cong V^{\ast} \otimes W$.
\begin{proof}
    This can be seen from the two spaces having the same dimension. We know $dim(V^{\ast} \otimes W) = dim V^{\ast} dim W = dim V dim W$. To complete the proof, if $\{v_1, \ldots, v_m\}$ and $\{w_1, \ldots, w_n\}$ are bases for $V$ and $W$, respesctively, then the collection of maps $g_{ij}: V \to W$ defined by $v_k g_{ij} = \delta_{ik} w_j$ (where $\delta$ is the Kronecker delta) is a basis for $Hom(V, W)$. So $dim Hom(V, W) = dim V dim W$.
\end{proof}
\end{prop}

\section{Tensor spaces}
\begin{defn}
    If we $V$ is any vector space, and if we define $V_1 := V$ and $V_{-1} := V^{\ast}$, then any tensor product $V_{\epsilon_1} \otimes \cdots \otimes V_{\epsilon_n}$ with each $\epsilon_i \in \{1, -1\}$, is isomorphic to $V^{\otimes p} \otimes (V^{\ast})^{\otimes q}$ for some $p, q \in \mathbb{N}$, by commutativity and associativity of the tensor product. We call such a product the \textbf{tensor space of type $(p, q)$} and denote it by $T_q^p(V)$. Elements of $T_q^p(V)$ are called \textbf{tensors of type $(p, q)$}. Such tensors are said to be \textbf{contravariant of degree $p$} and \textbf{covariant of degree q}. When $q = 0$, the tensors are said to be \textbf{contravariant}, and similary when $p = 0$ they are said to be \textbf{covariant}.
\end{defn}
\end{document}
