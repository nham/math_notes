\documentclass[a4paper,14pt]{article}
\title{Notes on Measure-Theoretic Probability}
\usepackage{amsmath, amssymb, amsthm}
\usepackage{tikz-cd}
\newtheorem*{prop}{Proposition}
\newtheorem*{corollary}{Corollary}
\newtheorem*{remark}{Remark}
\newtheorem*{defn}{Definition}
\begin{document}
\maketitle
\section{Algebras, $\sigma$-algebras, monotone classes}
An \textbf{algebra} of a set $X$ is any non-empty collection $\mathcal{A}$ of subsets of $X$ such that:

\begin{itemize}
    \item if $A \in \mathcal{A}$, then $X-A \in \mathcal{A}$
    \item if $A_1, \ldots, A_n \in \mathcal{A}$, then $\bigcup_1^n A_i \in \mathcal{A}$
\end{itemize}

A \textbf{$\sigma$-algebra} of a set $X$ is an algebra with the second condition replaced by this stronger one:

\begin{itemize}
    \item if $A \in \mathcal{A}$, then $X-A \in \mathcal{A}$
    \item if $(A_1, A_2, \ldots)$ is a sequence of sets in $\mathcal{A}$, then $\bigcup_1^{\infty} A_i \in \mathcal{A}$
\end{itemize}

In other words, an algebra is a non-empty collection of subsets closed under complement and finite union, and a $\sigma$-algebra is a non-empty collection of subsets closed under complement and \textit{countable} union.

It is straightforward to prove, using De Morgan's laws, that an algebra/$\sigma$-algebra is closed under finite/countable intersection.

A \textbf{monotone class} of a set $X$ is any non-empty collection $\mathcal{A}$ of subsets of $X$ such that:

\begin{itemize}
    \item if $(A_1, A_2, \ldots)$ is a sequence of sets in $\mathcal{A}$ with $A_n \subseteq A_{n+1}$ for all $n$, then $\bigcup_1^{\infty} A_i \in \mathcal{A}$
    \item if $(A_1, A_2, \ldots)$ is a sequence of sets in $\mathcal{A}$ with $A_n \supseteq A_{n+1}$ for all $n$, then $\bigcap_1^{\infty} A_i \in \mathcal{A}$
\end{itemize}

\begin{prop}
An algebra is a $\sigma$-algebra iff it is a monotone class.
\begin{proof}
Every $\sigma$-algebra is both an algebra and a monotone class. Conversely, if $\mathcal{A}$ is a both a monotone class and an algebra on $X$, then for any sequence $(A_1, A_2, \ldots)$ of subsets of $\mathcal{A}$, we can define $B_1 := A_1$, $B_n := B_{n-1} \cup A_n$. Then $B_n \subseteq B_{n+1}$ for all $n$, and since each $B_n$ is in $\mathcal{A}$ due to it being an algebra, we have $\bigcup_1^{\infty} B_i \in \mathcal{A}$ because it's a monotone class. But $\bigcup_1^{\infty} B_i = \bigcup_1^{\infty} A_i$, so this proves $\mathcal{A}$ is also a $\sigma$-algebra.
\end{proof}
\end{prop}

\begin{prop}
The intersection of any number of $\sigma$-algebras is a $\sigma$-algebra. The intersection of any number of monotone classes is a monotone class.
\end{prop}

\begin{defn}
For any set $X$ and any collection $\mathcal{C}$ of subsets of $X$, we denote by $\sigma(\mathcal{C})$ the intersection of all $\sigma$-algebras of $X$ that contain $\mathcal{C}$, called the \textbf{$\sigma$-algebra generated by $\mathcal{C}$}.
\end{defn}

\end{document}
